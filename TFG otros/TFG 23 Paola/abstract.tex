% !TeX encoding = ISO-8859-1

\begin{abstracts}

In this project, we study the solitary wave solution of the nonlinear Benjamin-Bona-Mahony Burgers equation based on a Galerkin method using quadratic B-splines.

In Chapter 1, we make a brief introduction about the origin of Soliton Theory, as well as, the emergence of the soliton term.

In Chapter 2, we introduce the equation and provide several historical reviews concerning to the origin of the Benjamin-Bona-Mahony equation.

In Chapter 3, we determined soliton solutions of the Benjamin-Bona-Mahony-Burgers equation by an exact method. Previously, we will define the concepts of travelling wave, solitary wave and soliton. Moreover, soliton solutions are determined for a particular case of the Benjamin-Bona-Mahony-Burgers equation.

In Chapter 4, based on the use of quadratic B-spline functions and Galerkin's method, we will obtain numerical solutions of the equation. Furthermore, we study the stability of the numerical method making use of the von Neumann theory.

In Chapter 5, we will consider two test problems. We will compare the numerical and exact solution, obtaining the errors by using the $L_{2}$ and $L_{\infty}$ norms. In addition, a fundamental soliton property is studied: the presence of conservation laws. Specifically, we will see that the mass, the momentum and the energy of the
waves analyzed in the numerical solution remain almost constant as time increases.

Finally, in the last chapter, the main conclusions which have been observed throughout the project are established.
\end{abstracts}
