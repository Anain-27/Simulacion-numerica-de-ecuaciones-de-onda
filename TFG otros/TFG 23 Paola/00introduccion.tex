% !TeX encoding = ISO-8859-1
\begin{savequote}[50mm]
Las matemáticas son la música de la razón
\qauthor{James Joseph Sylvester}
\end{savequote}

\chapter{Introducción}

\label{cha:Introduction}

En los ultimos años se han desarrollado innumerables contribuciones en el ámbito científico. Uno de los logros más significativos conseguidos en los últimos años, ha sido el desarrollo de la Teoría de Solitones. Estas ondas han llamado la atención de los matemáticos, físicos e ingenieros por su aplicabilidad en física. Los solitones se pueden entender como ondas no lineales, las cuales se propagan sin deformarse.\\

Aunque los fenómenos no lineales fueron estudiados en sistemas mecánicos, actualmente los encontramos en todo tipo de fenómenos naturales y sociales, tales como una ola aproximándose a la orilla. En la naturaleza los solitones juegan un papel importante en una variedad de fenómenos, como en fibras ópticas, sistemas geológicos, en canales, en ondas gigantes conocidas como Tsunamis, etc.\\

En el siglo XIX, el ingeniero escocés J.S. Russell dedicó muchos años a comprender este tipo de onda de transmisión conocida como solitón. J.S. Russell mientras se encontraba en el canal Union en Hermiston, Escocia, registró cierta peculiaridad en una onda creada por un bote en las aguas poco profundas. Observó que al detenerse el bote, la onda chocó con este, agitandose violentamente durante el choque pero recuperando su forma original y sin reducir su velocidad tras sobrepasarlo. Como podemos observar a continuación en las propias palabras de J.S. Russell:\\

$\ll$ \textit{Estaba observando el movimiento de un bote el cual era empujado rápidamente a lo largo de un estrecho canal por un par de caballos, cuando el bote se detuvo repentinamente, mas no así la masa de agua justo delante de la proa del bote en el canal, la cual, se había puesto en movimiento en un estado de violenta agitacion: repentinamente esta masa en agitación empezó a salir hacia delante con gran velocidad, tomando la forma de una larga elevación solitaria, redonda, suave y bien definida de una masa de agua, la cual continuó su curso a lo largo del canal aparentemente sin cambio de forma o disminución de velocidad. La seguí montado en un caballo, y aún la vi pasar a una razón de algunas ocho o nueve millas por hora, conservando su figura original algunos treinta pies de largo y entre un pie y pie y medio de altura. Pasado un tiempo, su altura gradualmente disminuyó, y después de seguirla una distancia de dos millas, la perdí en unos recodos del canal. Así que en el mes de agosto de 1834, fue mi primera oportunidad de encontrarme con ese singular y hermoso fenómeno, al cual he llamado Onda de Translación.} $\gg$\\

Durante varios años, esta conclusión de J.S. Russell fue cuestionada por reconocidos científicos de la época. No fue hasta 1965 cuando dos físicos-matemáticos
americanos, N. Zabusky y M. Kruskal, realizaron los trabajos pioneros en la obtención de soluciones numéricas de la ecuacion Korteweg-de Vries, demostrando la existencia de ondas solitarias que no sufrían deformación alguna, dándole el nombre a dichas soluciones no lineales de “solitones”.

