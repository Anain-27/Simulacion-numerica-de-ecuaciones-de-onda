% !TeX encoding = ISO-8859-1
\begin{resumen}

En este proyecto, estudiamos soluciones tipo onda solitaria de la ecuación no lineal de Benjamin-Bona-Mahony Burgers haciendo uso del método de Galerkin basado en B-spline cuadráticos.

En el Capítulo 1, hacemos una breve introducción al origen de la Teoría de Solitones, así como el surgimiento del término solitón.
En el Capítulo 2, introducimos la ecuación y aportamos varias reseñas históricas referentes al origen de la ecuación de Benjamin-Bona-Mahony.

En el capítulo 3, obtendremos soluciones tipo solitón de la ecuación de Benjamin-Bona-Mahony-Burgers. Previamente, definiremos los conceptos de onda viajera, onda solitaria y solitón. Por último, en este capítulo obtendremos la solución tipo solitón de la ecuación para un caso particular.

En el capítulo 4, basándonos en el uso de funciones B-spline cuadráticas y el método de Galerkin, obtendremos soluciones numéricas de la ecuación y estudiaremos la estabilidad de las mismas, haciendo uso de la teoría de von Neumann.

En el Capítulo 5, consideraremos dos problemas test. Realizaremos una comparación entre la solución numérica y exacta, obteniendo los errores mediante el uso de las normas $L_{2}$ y $L_{\infty}$. Además, se estudia una propiedad fundamental del solitón: la presencia de leyes de conservación. En concreto, veremos que la masa, el momento y la energía de las
ondas analizadas en la solución numérica permanecen casi constantes con el paso del tiempo.

Finalmente, se establecen las principales conclusiones que hemos observado a lo largo del trabajo.

\end{resumen}

