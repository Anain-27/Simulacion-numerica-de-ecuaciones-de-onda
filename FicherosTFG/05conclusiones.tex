% !TeX encoding = ISO-8859-1
% !TeX encoding = ISO-8859-1

\chapter{Conclusiones y proyectos futuros }
\label{Conclusiones}

A lo largo de este trabajo, hemos examinado la soluci�n num�rica de la ecuaci�n de onda mediante el m�todo de las diferencias finitas, observando en \ref{sec:soluciones}, como se obtienen buenos resultados para la misma. Hemos comparado distintas condiciones iniciales que cumplen con los par�metros obtenidos mediante el an�lisis de Von Neumann \ref{sec:VonNeumann} y el an�lisis energ�tico  \ref{energiadiscreta}, y hemos conseguido archivos de audio que lo corroboran.\\

Por otro lado, hemos logrado la implementaci�n de las soluciones de tipo onda viajera \ref{sec:solondaviajera}, de las que derivamos que la suma de las mismas describe inequ�vocamente el movimiento de la cuerda de nuestro esquema.\\

A continuaci�n hemos representado la relaci�n entre la condici�n CFL y la inestabilidad de las soluciones, observando como peque�as variaciones en ella generan dicha inestabilidad de forma acelerada.\\

Por �ltimo se ha visualizado como si no se respeta la ley de la conservaci�n de la energ�a, es decir, se incumplen las condiciones de contorno dadas en la secci�n \ref{energiadiscreta}, la soluci�n num�rica va perdiendo energ�a hasta pararse.(NO ESTOY SEGURA TENGO QUE COMPROBARLO CUANDO TERMINE LA SECCI�N) Esto puede comprobarse en la secci�n \ref{sec:ComparacionEnergia}(COMPLETAR CUANDO EST� HECHO)\\

En definitiva, hemos podido realizar un estudio general de la ecuaci�n de onda en derivadas parciales, abordando desde los conceptos fundamentales hasta la comprobaci�n de como las condiciones te�ricas encontradas afectan la soluci�n num�rica de la misma.\\

Este estudio ser�a f�cilmente ampliable tomando varios enfoques,

\begin{itemize}
	\item Utilizando argumentos similares a los encontrados en \cite{Optimal}, podr�amos hablar de la controlabilidad y observabilidad de nuestro esquema, en lugar de en un ambiente semi-discreto como el que propone Enrique Zazua.
	\item Se podr�a enriquecer el estudio comparando diferentes an�lisis usados asiduamente en el m�todo de las diferencias finitas.
	\item Tambi�n ser�a interesante el uso de m�todos de resoluci�n diferentes, como por ejemplo los m�todos espectrales.
	\item Un enfoque menos te�rico que se podr�a dar es, la utilizaci�n de los archivos .wav sintetizados para la creaci�n de m�sica, ya que hemos visto, que dependiendo de las condiciones iniciales, la forma de la onda obtenida cambia, de manera similar a como lo hace al tocar la misma nota en dos instrumentos diferentes.
\end{itemize}







% ----------------------------------------------------------------------

 