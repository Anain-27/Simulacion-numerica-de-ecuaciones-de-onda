% !TeX encoding = ISO-8859-1

\chapter{Conclusiones}

A lo largo del trabajo se han presentado numerosos modelos que utilizan la SVM como herramienta para clasificar un conjunto de datos en dos clases, as� mismo, se ha puesto especial �nfasis tanto en la construcci�n de modelos que reduzcan en n�mero de atributos seleccionados como en el desarrollo de procedimientos que permitan resolver dichos modelos de forma eficiente. Se ha visto que los modelos presentados son bastante efectivos a la hora de clasificar, lleg�ndose a obtener una ACC y una AUC por encima del 90\% en muchos casos, al mismo tiempo que se reduc�an considerablemente el n�mero de atributos a valorar. La reducci�n de caracter�sticas proporciona numerosas ventajas, por ejemplo, al analizar datos m�dicos, especificar cu�les son los atributos m�s influyentes en la clasificaci�n y que permiten a los especialistas centrarse en determinados par�metros, tanto para la prevenci�n de la enfermedad estudiada, como en la investigaci�n m�dica. Esto mismo es aplicable a otros campos, como en estudios de clientes bancarios, en modelos sociales, etc�tera.

En vista de los resultados obtenidos en el Cap�tulo \ref{cap:5}, ser�a bastante interesante seguir depurando la cota de M, lo que permitir�a  resolver los modelos de Ramp Loss en muestras de tama�o superior.  As� mismo, ser�a interesante adaptar las nuevas estrategias de resoluci�n expuestas para los modelos de Ramp Loss a los modelos Hard Margin Loss.

Por otra parte, aunque durante todo el trabajo se han utilizado la SVM para divisi�n del conjunto de datos en dos clases, esta puede ser utilizada para dividir los datos en $N$ clases, ejecut�ndolas de manera repetitiva. As� mismo, en la literatura podemos encontrar t�cnicas que evitan repetir el procedimiento $N$ veces, dando lugar a la (M-SVM), es decir, a la m�quina de vectores soporte multicategor�a, ver \cite{multicategorySVM}. Una l�nea de investigaci�n interesante ser�a estudiar los modelos presentados para la (M-SVM) e incluir las herramientas de selecci�n de caracter�sticas vistas a lo largo del trabajo.



%Remarcar el hecho SVM puede aplicarse a varias categor�as aplicadno la SVM varias veces e incluso 

%es posible generalizarlo con la SVM a trav�s de la psi-function aunque en general no es un porblema de minimizaci�n c�ncava, tb es posible utilizar esta herramienta para muchas categprias (tengo tres art�culos descargados y en el proyecto fin de m�ster tb hay cosas)


% ----------------------------------------------------------------------

 