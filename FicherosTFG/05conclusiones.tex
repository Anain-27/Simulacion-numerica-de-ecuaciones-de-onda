% !TeX encoding = ISO-8859-1
% !TeX encoding = ISO-8859-1

\chapter{Conclusiones y proyectos futuros }
\label{Conclusiones}
A lo largo de este trabajo, hemos introducido conceptos matem�ticos, tales como los espacios de Lebesgue y el producto escalar, los hemos generalizado para el caso discreto, en un mallado definido en ambiente discreto. Ha sido fundamental definir distintos tipos de aproximaciones (progresivas, regresivas y centradas), que usadas adecuadamente nos llevan a esquemas num�ricos estables\\

Se ha contextualizado la ecuaci�n en derivadas parciales de onda, utilizando una perspectiva f�sica. Tambi�n se han analizado ciertas soluciones de la ecuaci�n, y la Ley de la conservaci�n de energ�a que satisface la misma.\\

Posteriormente, nos centramos en aproximar la ecuaci�n de onda mediante el m�todo de diferencias finitas. Luego, examinamos las caracter�sticas del esquema de aproximaci�n obtenido, constatamos la estabilidad del esquema bajo ciertas condiciones, la consistencia y la ley de conservaci�n de energ�a discreta.\\

Proseguimos implementando en Python el esquema modelado, realizando diversas simulaciones. Abordamos la influencia de la frecuencia en la onda sonora obtenida y la hemos sintetizado de manera num�rica.\\


Por otro lado, hemos logrado la implementaci�n de las soluciones de tipo onda viajera \ref{sec:solondaviajera}, reiterando que la suma de las mismas describe inequ�vocamente el movimiento de la cuerda.\\

A continuaci�n, observamos el comportamiento de las aproximaciones, a comparaci�n de la soluci�n exacta. Variamos las condiciones obtenidas mediante el an�lisis de la estabilidad, representado la relaci�n entre la condici�n CFL y la inestabilidad de las soluciones. Observando como su incumplimiento conduce a aproximaciones incorrectas, que llevan que presentan multitud de oscilaciones esp�reas.\\

En definitiva, hemos podido realizar un estudio general de la ecuaci�n de onda en derivadas parciales, abordando desde los conceptos fundamentales, hasta la comprobaci�n de como las condiciones te�ricas son fundamentales para obtener una correcta soluci�n num�rica de la misma.\\

Este estudio ser�a f�cilmente ampliable tomando varios enfoques,

\begin{itemize}
	\item Utilizando argumentos similares a los encontrados en \cite{Optimal}, podr�amos hablar de la controlabilidad y observabilidad de nuestro esquema, en lugar de en un ambiente semi-discreto como el que propone Enrique Zuazua.
	\item Se podr�a enriquecer el estudio comparando diferentes esquemas usados asiduamente en el m�todo de las diferencias finitas, \citep{Russell, Optimal}.
	\item Tambi�n ser�a interesante estudiar si es adecuado el uso de m�todos de resoluci�n diferentes, como por ejemplo los m�todos espectrales, o elementos finitos \citep{Stig}.
	\item Considerar ecuaciones de ondas en dominios bidimensionales, como la piel de un tambor que vibra al ser golpeado por una baqueta.
	\item Un enfoque m�s aplicado que podr�amos abordar es la utilizaci�n de los archivos .wav sintetizados para la creaci�n de m�sica, ya que hemos visto, que dependiendo de las condiciones iniciales, la forma de la onda obtenida cambia, de manera similar a como lo hace al tocar la misma nota en dos instrumentos diferentes.
	
\end{itemize}







% ----------------------------------------------------------------------

 