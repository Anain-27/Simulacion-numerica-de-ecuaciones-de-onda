% !TeX encoding = ISO-8859-1
\chapter{Introducci�n}
\label{cha:Introduction}

Los avances a pasos agigantados de la computaci�n y la tecnolog�a de los �ltimos a�os, han logrado que matem�ticas, y en particular en el campo de las ecuaciones diferenciales, hayan dado un vuelco.\\

En situaciones, en las que anteriormente ten�amos que desistir o conformarnos con soluciones parciales o en marcos espec�ficos ahora podemos hacer uso de los m�todos num�ricos para conseguir una aproximaci�n a la realidad cada vez m�s exacta.\\

En este trabajo utilizaremos est�s t�cnicas, para observar la resoluci�n de las ecuaciones en derivadas parciales de onda. \\

La belleza de obtener num�ricamente estas soluciones es que podremos manejarlas para obtener, gr�ficas, y distintos tipos de archivos, cada vez m�s exactos conforme la potencia de los dispositivos va mejorando.\\

Nosotros nos centraremos en dar una visi�n general de la resoluci�n de la ecuaci�n de onda, en el cap�tulo \ref{cha:preliminares} nos adentraremos en los conceptos previos necesarios para la comprensi�n del trabajo, como los Espacios de Lebesgue, el concepto de producto interior, o el mallado que utilizaremos para definir nuestro esquema num�rico.\\

En el cap�tulo siguiente, \ref{cha:Comenzamos}, introduciremos la ecuaci�n de onda desde un marco f�sico y realizaremos una clasificaci�n de la misma. Posteriormente nos centraremos en un tipo concreto de soluciones, llamadas de onda viajera. Por �ltimo, analizaremos la energ�a de la ecuaci�n y veremos en que condiciones se cumplir� la Ley de la conservaci�n de la energ�a. \\

Una vez tenemos todos los cimientos construidos, en el cap�tulo \ref{cha:Comparativa}, haremos uso de uno de los m�todos m�s utilizados para la resoluci�n num�rica, el m�todo de las diferencias finitas. Continuando el cap�tulo, analizaremos la consistencia y estabilidad de las soluciones obtenidas. Acabaremos trasladando tanto las soluciones de tipo onda viajera como la conservaci�n de la energ�a al �mbito discreto.\\


Concluiremos el grueso del trabajo llevando todos los avances hechos  te�ricamente a la practica. Esto lo realizaremos mediante el script de Python, este se puede encontrar en el Anexo \ref{cha:Anexo} y en el Github del trabajo, . En el cap�tulo \ref{cha:Comparativa}, comparamos las soluciones obtenidas, en varios sentidos, desde cambiando las condiciones iniciales del sistema, hasta varios de las condiciones obtenidas en los cap�tulos anteriores.\\


Como cierre, en el apartado de conclusiones \ref{Conclusiones}, sintetizaremos los hallazgos obtenidos y presentaremos sugerencias para seguir explorando en esta �rea de estudio.





