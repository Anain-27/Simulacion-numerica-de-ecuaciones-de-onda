% !TeX encoding = ISO-8859-1

\chapter{Definici�n de la ecuaci�n de onda}
\label{cha:Comenzamos}
!!!!Explicar porque modela una onda\\
Comenzaremos dando una definici�n general de la ecuaci�n de ondas en $\mathbb{R}^n$, con $n \geq 1$



\begin{equation}
	u_{tt}- c^2 \bigtriangleup u = f , x\in \mathbb{R}^n, t\in\mathbb{R}
\end{equation}

Con $f$ es una funci�n de valores reales, $c>0$ constante de propagaci�n y el operador 
\begin{equation}
	\bigtriangleup u = \sum_{j=1}^{N}\frac{\delta^2u}{\delta x^2_j}
\end{equation}

que es conocida como el Laplaciano. En este tipo de ecuaciones la variable $t$ denota el tiempo transcurrido y las variables $x_i, i=1,...,n$ denotan la posici�n de la onda en cada direcciones del espacio $\mathbb{R}^n$. \\

Podemos observar que son lineales y siempre que $f\not\equiv 0$ la ecuaci�n ser� no homog�nea, es decir tendr� termino independiente.\\




Comenzaremos hablando de las ecuaci�n de onda de una dimensi�n puesto que esas son las mas sencillas de trabajar, luego intentaremos generalizar los resultados obtenidos\\
La ecuaci�n se definir� entonces de la forma:
\begin{equation}
 u_{tt}=c^2u_{xx}
\end{equation}
Para poder tratar con este tipo de ecuaciones de manera mas sencilla usualmente las clasificamos, en el�pticas, hiperb�licas y parab�licas, para ellos tendremos que ponerla en forma est�ndar. Aunque se pueden clasificar y estandarizar siempre nosotros por simplicidad lo haremos solo para el segundo orden.\\
Una ecuaci�n de segundo orden en forma est�ndar ser�a como sigue:
\begin{equation}
	Au_{tt}+Bu_{tx}+Cu_{xx}+Du_{t}+Eu_{x}+Fu= G
\end{equation}
donde, $A,B,C,D,E,F,G$ son constantes o funciones de variables t y x.\\
Clasificaremos dependiendo de  las ecuaciones caracter�sticas relativas a la EDP:

\begin{definicion}[label={clasificacion},nameref={Title or anything else}]{Clasificaci�n de las EDP de segundo orden}	Si en todos los puntos $(x,y)$ de una regi�n $W\in \mathbb{R}^2$ se cumple que:
	\begin{itemize}
		\item $B^2(x,y)-4A(x,y)C(x,y)>0$, entonces la EDP (n*) se dice hiperb�lica.
		\item $B^2(x,y)-4A(x,y)C(x,y)=0$, entonces la EDP (n*) se dice parab�lica.
		\item $B^2(x,y)-4A(x,y)C(x,y)<0$, entonces la EDP (n*) se dice el�ptica.
		
	\end{itemize}
\end{definicion}



Pongamos ahora nuestra ecuaci�n de esa forma:\\
Vemos que $A=1, C=-c^2, G=f$ y cada una de las dem�s variables es 0. Ahora tenemos $B^2-4AC=4c^2>0$ y por lo tanto nuestra ecuaci�n es hiperb�lica.\\
Este tipo de clasificaci�n de las ecuaciones se usa para separar los distinto m�todos num�ricos que se pueden usar, ya que estos suelen diferir dependiendo de la misma.


