% !TeX encoding = ISO-8859-1

\chapter{Anexo}

En este cap�tulo anexaremos el c�digo realizado en Python.\\

\section{Resoluci�n num�rica}
Comenzamos implementando el esquema \eqref{esquemacompletoonda} y as� resolviendo la ecuaci�n de ondas en ese contexto.
 \begin{verbatim}
 	import numpy as np
 	import time
 	import os as os
 	from shutil import rmtree
 	import matplotlib.pyplot as plt
 	
 	#Funciones necesarias
 	
 	# Para inicializar los valores iniciales que dependen de la derivada lo haremos con una aproximaci�n progresiva del tiempo
 	def g(x):
 	return 0
 	
 	
 	# Condici�n de contorno en el eje de posici�n tipo seno
 	def f(x):
 	return np.sin(np.pi * x)  # Tener cuidado con la funci�n por si se sale del rango
 	
 	
 	
 	# Definimos una funci�n para usar en el m�todo expl�cito y poder cambiarlo de manera r�pida si hace falta
 	def Explicito(x1, x2, x3, x4,mu):  # Si queremos hacerlo en un file independiente tendremos que a�adir el mu
 	return  mu ** 2 * (x1 + x2) + 2 * (1- mu ** 2) * x3 - x4
 	
 	
 	start = time.time()
 	print("Vamos a hacer una resoluci�n num�rica de la ecuaci�n de onda u_tt-c^2u_xx=0")
 	print("Introduce c:")
 	c= float(input())
 	print("Introduce el nombre del directorio donde quieres guardar los archivos:")
 	directorio=input()
 	
 	
 	#Creamos una carpeta en la que se guardar�n los datos
 	home = 'C:\\Users\\Ana Cuevas de C�zar\\PycharmProjects\\pythonProject2'
 	directoriofinal=home+'\\'+directorio+'\\'
 	#Si existe el directorio lo borra
 	try:
 	rmtree(directoriofinal)
 	except:
 	print('No exist�a el directorio, lo creamos')
 	os.mkdir(directoriofinal)
 	
 	
 	#Constantes necesarias
 	l=1
 	tmax= 1
 	m = int(l*3)# Constante representativa del n�mero de trozos en los que separamos el espacio
 	h=l/m  #Cambiamos la h para que sea entera
 	
 	k =h/c #Lo tomamos de esta forma para ganar en exactitud pagina 133 numerical sound synthesis
 	#k =h/c * 0.01
 	#k =h/c * 0.1
 	#k =h/c * 0.5
 	#k =h/c * 0.9
 	n= int(tmax/k)
 	
 	#Comprobaci�n de que es entero!
 	mu = c * k / h
 	
 	if mu>1 :
 	raise Exception('Mu es mayor que 1')
 	print(f'n{n}, c{c}, k{k}, h{h}, m{m}, mu{mu}')
 	# Creamos un mallado de puntos en los que aproximaremos la soluci�n
 	x= np.linspace(0,l,num=m +1)
 	t= np.linspace(0,tmax,num=n +1)
 	
 	
 	
 	# Creamos la matriz u de soluciones
 	# u(x,t)
 	u = np.empty((m + 1, n + 1), float)
 	
 	
 	# A�adimos antes que nada los nodos conocidos por las condiciones de contorno
 	
 	u[:, 0] = f(x)
 	
 	u[0, :] = np.zeros(n + 1)
 	
 	u[m, :] = np.zeros(n + 1)
 	aux=[0]
 	# Para inicializar los valores iniciales que dependen de la derivada lo haremos con una aproximaci�n progresiva del tiempo
 	for i in range(1, m):
 	u[i, 1] = u[i, 0] + k * g(x[i])
 	aux.append(g(x[i]))
 	
 	aux.append(g(x[m]))
 	
 	print(u[:,1])
 	# Por otro lado si nuestra funci�n inicial f es C^2 podemos hacerlo de esta otra forma
 	# Para tener un error bastante mas peque�o
 	# for i in range(1,m):
 	# u[i,1]=u[i,0]+k*g(x[i])+ (c^2*k*2/2)f''(x[i])        Meter la derivada de f
 	
 	for j in range(2, n + 1):
 	u[1:m, j] = Explicito(u[2:m+1,j-1], u[0:m-1,j-1], u[1:m,j-1], u[1:m,j-2],mu)
 	
 	end = time.time()
 	
 	
 	plt.plot(x,aux,'k')
 	plt.xlabel('x')
 	plt.ylabel('g(x)')
 	plt.title(f'G')
 	plt.savefig(directoriofinal+'prueba g.png')
 	plt.close()
 	
 	print(f'Ha tardado {np.floor(end-start)} segundos.')
 	np.save(directoriofinal+'u', u)
 	np.save(directoriofinal+'x', x)
 	np.save(directoriofinal+'t', t)
 	np.save(directoriofinal+'constantes', [c, m,h,n,k,mu,tmax] ) #Meter en cons las constantes que necesitemos para usarlas luego en el guardado
 \end{verbatim}
 Al final guardamos los datos obtenidos para su posterior procesado.
 
\section{Wave}
Creo ahora un archivo de audio a partir de los datos anteriores y lo guardo en un archivo de sonido .wav
\begin{verbatim}
	import numpy as np
	from scipy.io.wavfile import write
	
	
	print("Vamos a guardar nuestros datos en formato wav")
	print("Introduce el nombre de la carpeta:")
	dir= str(input())
	
	
	#Creamos el directorio
	directorio='C:\\Users\\Ana Cuevas de C�zar\\PycharmProjects\\pythonProject2\\'+dir+'\\'
	
	
	#A�adimos los datos
	#A�adimos los datos
	datos = np.load(directorio+'u.npy', mmap_mode='r')
	t = np.load(directorio+'t.npy', mmap_mode='r')
	x = np.load(directorio+'x.npy', mmap_mode='r')
	[f, m1,h1,n1,k1,mu,tmax]= np.load(directorio+'constantes.npy', mmap_mode='r')
	
	m1=int(m1)
	n1=int(n1)
	
	#Constantes necesarias
	sample_rate = 44100
	n2 = int(tmax*sample_rate)
	t_necesario= np.linspace(0,tmax,n2 +1)
	
	
	# Escogiendo un punto en particular
	# Comprobamos donde est� el punto de mayor amplitud en el instante inicial y en ese punto es donde veremos como se mueve la onda.
	pos_readout= np.where(datos[:,0]==max(datos[:,0]))[0][0]
	
	if pos_readout==0:
	pos_readout= 50
	print('Cambio la posicion a 0.5')
	
	wave_table = datos[pos_readout] # Elijo un punto en el espacio en el que veremos el movimiento de la cuerda
	wave_table2 = datos[pos_readout+1] # Elijo un punto en el espacio en el que veremos el movimiento de la cuerda
	output = np.interp(t_necesario,  t, wave_table)
	
	
	# Sumando todos los puntos en espacio que tenemos.
	#wave_table = np.zeros(n1 + 1)
	#for i in range(0, m1+1):
	#wave_table += datos[i, :]
	
	#wave_table= wave_table/(m1+1)
	#output = np.interp(t_necesario,  t, wave_table)
	
	
	#Vamos a modificar los limites para que suene a un nivel de volumen adecuado, para ello teben ser en torno al orden e-1 o e-2
	#gain= 0
	#amplitud = 10** (gain/20)
	#output *=amplitud
	
	
	# Para que al principio y final se escuche mas suave
	#output = fade(output)
	
	#Guardamos
	write(directorio+f'Sonido.wav',sample_rate,output)
\end{verbatim}

\section{Imagenes}
Creamos y guardamos las im�genes de la posici�n de la cuerda en varios instantes de tiempo y de los cambios de posici�n de la onda en un punto del eje $X$ durante un periodo de tiempo.

\begin{verbatim}
	import numpy as np
	import matplotlib.pyplot as plt
	from os import mkdir
	from shutil import rmtree
	
	print("Vamos a guardar nuestras imagenes")
	print("Introduce el nombre de la carpeta:")
	dir= str(input())
	
	
	#Creamos los directorios
	directorio='C:\\Users\\Ana Cuevas de C�zar\\PycharmProjects\\pythonProject2\\'+dir+'\\'
	cuerda=directorio + 'Imagenes Cuerda\\'
	onda= directorio+ 'Imagenes Onda\\'
	try:
	rmtree(cuerda)
	rmtree(onda)
	except:
	print('No exist�an las imagenes, las creamos')
	mkdir(onda)
	mkdir(cuerda)
	
	
	
	#A�adimos los datos
	datos = np.load(directorio+'u.npy', mmap_mode='r')
	t = np.load(directorio+'t.npy', mmap_mode='r')
	x = np.load(directorio+'x.npy', mmap_mode='r')
	[f, m,h,n,k,mu,tmax]= np.load(directorio+'constantes.npy', mmap_mode='r')
	
	n=int(n)
	m=int(m)
	
	#Vamos a pintar las imagenes de a posici�n de la cuerda en instantes selecionados
	plt.plot(x, datos[:, 0], 'k')
	plt.ylim(-1,1)
	plt.xlabel('x')
	plt.ylabel('u')
	plt.title(f'Cuerda en el instante {t[0]},\nmin={min(datos[:,0])}, max={max(datos[:,0])}')
	plt.savefig(cuerda+f'Cuerda_instante_{0}.png')
	plt.close()
	
	for j in range(1, n + 1):
	#if j % int(10*n/f)==0:
	if j==j:
	plt.plot(x, datos[:, j], 'k')
	plt.ylim(-1,1)
	plt.xlabel('x')
	plt.ylabel('u')
	plt.title(f'Cuerda en el instante {t[j]},\nmin={min(datos[:,j])}, max={max(datos[:,j])}')
	plt.savefig(cuerda+f'Cuerda_instante_{j}.png')
	plt.close()
	
	
	#Pintamos la imagen de la onda en los 1000 primeros instantes en cada punto
	for j in range(0, m + 1):
	plt.plot(t[0:1000], datos[j,0:1000],'r')
	plt.plot(t[0:1000], datos[j,n-999:n+1],'b')
	plt.xlabel('t')
	plt.ylabel('u')
	plt.title(f'Movimiento del punto {x[j]} en el espacio')  #axis tight
	plt.savefig(onda+f'Punto_{j}.png')
	plt.close()
\end{verbatim}

% ----------------------------------------------------------------------

 