% !TeX encoding = ISO-8859-1

\chapter{Tipos de soluciones}
\label{cha:preliminares}

\section{Sus tipos}

Las ecuaciones hiperb�licas en general no se pueden resolver con m�todos tradicionales exactos de resoluci�n, (ejemplos de m�todos). Para que se pudiesen resolver de manera exacta tendr�an que cumplir unas condiciones muy concretas y en la practica no nos resultar�an muy �tiles, al no modelar la realidad. Esto nos lleva a la creaci�n de m�todos num�ricos que consiguen resolverlas de una manera no exacta.
En la actualidad para las ecuaciones hiperb�licas el m�todo mas usado es el de los vol�menes finitos, y por ello comenzaremos por el mismo.

\section{M�todo de los vol�menes finitos}

El m�todo de los vol�menes finitos 




% ----------------------------------------------------------------------

